\section{Usage Examples}

\subsection{Full-Duplex Communication}
The following example demonstrates full-duplex SPI communication in mode 1 (CPOL=0, CPHA=1):

\begin{verbatim}
// 1. Configure Master and Slave for mode 1
write(MASTER_CTRLB, 0x01);    // Set SPI mode 1 (CPOL=0, CPHA=1)
write(SLAVE_CTRLB, 0x01);     // Set same mode on slave

// 2. Prepare data for transmission
write(MASTER_DATA, 0xA5);     // Master will send 0xA5
write(SLAVE_DATA, 0x3C);      // Slave will send 0x3C

// 3. Enable Master and Slave
write(MASTER_CTRLA, 0x21);    // Enable Master, Enable SPI
write(SLAVE_CTRLA, 0x01);     // Enable Slave

// 4. Wait for transmission to complete
while (!(read(MASTER_INTFLAGS) & 0x80)); // Wait for RXCIF flag

// 5. Read received data
uint8_t master_rx = read(MASTER_DATA);  // Should be 0x3C
uint8_t slave_rx = read(SLAVE_DATA);    // Should be 0xA5
\end{verbatim}

\subsection{Changing Bit Order}
The following example demonstrates LSB-first transmission:

\begin{verbatim}
// 1. Configure Master for LSB-first transmission
write(MASTER_CTRLB, 0x80);    // Set DORD bit (LSB first)

// 2. Prepare data
write(MASTER_DATA, 0x81);      // Binary: 10000001

// 3. Enable Master
write(MASTER_CTRLA, 0x21);     // Enable Master, Enable SPI

// 4. Expected transmission order:
//    Bit 0 (1) -> Bit 1 (0) -> ... -> Bit 7 (1)
\end{verbatim}

\subsection{Buffer Mode Operation}
The following example demonstrates buffered transmission:

\begin{verbatim}
// 1. Enable buffer mode (must be done before writing data)
write(MASTER_CTRLB, 0x80);    // Enable buffer mode

// 2. Prepare multiple data bytes
write(MASTER_DATA, 0x12);     // First byte
write(MASTER_DATA, 0x34);     // Second byte (buffered)
write(MASTER_DATA, 0x56);     // Third byte (buffered)

// 3. Enable Master with interrupt
write(MASTER_INTCTRL, 0x40);  // Enable TX complete interrupt
write(MASTER_CTRLA, 0x21);    // Enable Master, Enable SPI

// 4. Check buffer status
while (!(read(MASTER_INTFLAGS) & 0x40)); // Wait for TXCIF flag

// 5. Transmission continues automatically with buffered data
\end{verbatim}

\subsection{Normal Reception with Interrupts}
The following example demonstrates reception with interrupt:

\begin{verbatim}
// 1. Configure reception interrupt
write(MASTER_INTCTRL, 0x01);  // Enable RX complete interrupt

// 2. Enable Master
write(MASTER_CTRLA, 0x21);    // Enable Master, Enable SPI

// 3. In interrupt service routine:
void SPI_ISR() {
    if (read(MASTER_INTFLAGS) & 0x80) {  // Check RXCIF
        uint8_t data = read(MASTER_DATA);
        // Process received data...
        write(MASTER_INTFLAGS, 0x80);    // Clear flag
    }
}
\end{verbatim}

\subsection{Daisy Chain Configuration}
The following example demonstrates daisy-chained SPI devices:

\begin{verbatim}
// 1. Configure all devices (master + 2 slaves)
write(MASTER_CTRLA, 0x21);    // Enable Master
write(SLAVE1_CTRLA, 0x01);    // Enable Slave 1
write(SLAVE2_CTRLA, 0x01);    // Enable Slave 2

// 2. Send data through the chain
write(MASTER_DATA, 0xAA);     // First byte for slave 1
write(MASTER_DATA, 0xBB);     // Second byte for slave 2

// 3. Wait for complete transmission
delay(SPI_TRANSFER_TIME * 3); // Wait for 3 full transfers

// 4. Read received data
uint8_t slave1_rx = read(SLAVE1_DATA);  // Should be 0xAA
uint8_t slave2_rx = read(SLAVE2_DATA);  // Should be 0xBB
uint8_t master_rx = read(MASTER_DATA);  // Returns data from slave 2
\end{verbatim}
