\section{Simulation}

\subsection{Tests}
The test bench generates a number (default is 2) configurations of the
SPI that are highly randomized. The user can run the full test-suite  
with n-configurations using "sbt "test" -DtestName="regression"".

User can also run individual tests or test groups by substituting "regression" with
the relevant test name. A description of the tests is below:

\begin{table}[h]
  \resizebox{\textwidth}{!}{
  \centering
  \begin{tabular}{|c|c|c|c|}
      \hline
      \rowcolor{dark-gray}  % Dark grey background for header row
      \textcolor{white}{\textbf{Test Group Name}} & \textcolor{white}{\textbf{Test Name}} & \textcolor{white}{\textbf{Test Type}} & \textcolor{white}{\textbf{Test Description}} \\ \hline
      transmitTests & masterMode & Directed & Configures SPI as Master and Tests Transmit w/ Random Data \\ \hline
      transmitTests & slaveMode & Directed & Configures SPI as Slave and Tests Transmit w/ Random Data \\ \hline
      transmitTests & fullDuplex & Random & Configures one SPI as Master, other as Slave. Tests transmission w/ Random Data Across all 4 CPOL/CPHA Modes \\ \hline 
      transmitTests & bitOrder & Directed & Should Transmit and Recieve Random Data Correctly in MSB and LSB modes \\ \hline  
      clockTests & prescaler & Directed & Clock Speed Test for All Prescaler Values \\ \hline
      clockTests & doubleSpeed & Directed & Clock Speed Test with Double Speed Enabled \\ \hline
      interruptTests & txComplete & Directed & Test Transmission Complete Interrupt Flag \\ \hline
      interruptTests & wcolFlag & Directed & Check if Write Collision Flag is Triggered in Normal Mode \\ \hline
      interruptTests & dataEmpty & Directed & Check Data Register Empty Flag Triggers in Buffer Mode \\ \hline
      interruptTests & overFlow & Random & Check Buffer Mode Operation still Continue with Overflow, and Overflow Flag is Set \\ \hline
      modeTests & bufferTx & Directed & Verify Data is Transmitted and Recieved Correctly in Buffer Mode \\ \hline
      modeTests & normalRx & Directed & Check Recieve Register Operation in Normal Mode \\ \hline
      modeTests & daisyChain & Random & Check SPI Operation w/ 1 Master and 2 Slaves in Daisy Chain in Normal Mode \\ \hline
      modeTests & daisyChainBuffer & Random & Check SPI Operation w/ 1 Master and 2 Slaves in Daisy Chain in Buffer Mode \\ \hline
    \end{tabular}
  }
  \caption{Test Suite}
\end{table}


\subsection{Code coverage}
All inputs and outputs are checked to insure each toggle at least once. An error
will be thrown in case any port fails to toggle.

The only exception are the \emph{almostEmptyLevel} and \emph{almostFullLevel}
which are intended to be static during each simulation. These signals are
excluded from coverage checks.

\subsection{Running simulation}

Simulations can be run directly from the command prompt as follows:

\begin{verbatim}
  $ sbt "test"
\end{verbatim}

or from make as follows:

\texttt{\$ make test}

\subsection{Viewing the waveforms}

The simulation generates an FST file that can be viewed using a waveform viewer. The command to view the waveform is as follows:
\begin{verbatim}
  $ gtkwave ./out/test/Gpio.fst
\end{verbatim}
